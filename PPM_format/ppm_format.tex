\documentclass{article}
\usepackage{float, hyperref, graphicx, listings,textcomp, subcaption}
\title{Notes on PPM format}
\author{Amritpal Singh}
\date{4 March 2023}
\begin{document}
\maketitle

\newpage
\tableofcontents
\vspace{400px}
Any errors in this document are most likely a product of my carelessness, 
please report any errors if you find them at
\href{mailto:sysgrammer@protonmail.com}{sysgrammer@protonmail.com}

\newpage
\section{Introduction}
There are my notes on the PPM image file format after reading the wikipedia page
and the netpbm's specification on PPM. PPM stands for Portable Pixel Map (
there also exists PBM -- Portable Bit Map as well).

\section{Header}
The header of a PPM file is always in ASCII. It is described as follows in form
of a list:
\begin{enumerate}
    \item P$x$ where $x \in X = \{3, 6\}$.
    \item Whitespace.\footnotemark[1]
    \item Width (in base 10 or decimal).
    \item Whitespace.\footnotemark[1]
    \item Height (in base 10 or decimal).
    \item Whitespace.\footnotemark[1]
    \item Maximum Color Value (mcv) where $0 < mcv < 65536$.
    \item Whitespace.\footnotemark[1]
\end{enumerate}


Afterwards raster data is followed, which we will describe now.

\section{Raster Data}

Each pixel of a P3 or P6 image is defined as a triplet of red, green, blue
values. A P3 type PPM image is known as a Plain PPM file because all of its
pixel data is in ASCII as well, in contrast to P6 PPM which is known as Binary
PPM where the header in ASCII but the raster data is in binary form.

Note that, line following \verb|#| is a comment, don't use this in binary mode
as it may skew the images to the right.

\subsection{Binary PPM (P6)}

In binary form, each of these values is represented in 1 or 2 bytes.
If the $mcv$ is less than 256, then each value is of 1 byte, otherwise 2 bytes.

The Raster data following the $(8)^{th}$ whitespace character is in Height 
rows. The rows are ordered from top to bottom and each row is Width pixels wide
, the pixels have been defined as specified above where Pixels in a row are
 from left-to-right ordering. 
\\
A row of an image is horizontal and the columns are vertical. There is no actual
separation required between each pixel in a row unike the "Plain" PPM file 
described ahead.

\subsection{Plain PPM (P3)}

The Plain PPM format is fairly simple, instead of having binary data it
contains plain ascii data where each pixel is a decimal value as ASCII.
The following part highlights key differences between the two:
\begin{itemize}
    \item There is exactly one image in one file.
    \item Magic number is P3 instead of P6.
    \item Each pixel sample/value is represented as ASCII in 
    decimal/base10 of arbitrary size.
    \item Each pixel sample/value has a whitespace\footnotemark[1] before and
    after it. There must atleast be one character of whitespace\footnotemark[1]
    but there is no maximum. There is not exact separation between pixel --- 
    just the required separation between blue sample from one pixel and the red
    sample from the next pixel.
    \item No line should be longer than 70 characters.
\end{itemize}

\subsection{P3 Example}
Here is an example of a p3 type ppm image

\begin{figure}[H]
\centering
\begin{lstlisting}
P3
# feep.ppm
4 4
15
0  0  0    0  0  0    0  0  0   15  0 15
0  0  0    0 15  7    0  0  0    0  0  0
0  0  0    0  0  0    0 15  7    0  0  0
15  0 15   0  0  0    0  0  0    0  0  0
\end{lstlisting}
\caption[short]{Nicely formatted P3 PPM}
\end{figure}

It does not matter whether you write your data in actual rows of 4x4
but that each pixel's individual sample and the pixels are spearated
by a whitespace\footnotemark[1]. You can even write one pixel per line.

An Example would be as follows:

\begin{figure}[H]
    \centering
\begin{lstlisting}
P3
# feep.ppm
4 4
15
0  0  0
0  0  0 
0  0  0 
15  0 15
0  0  0 
0 15  7 
0  0  0 
0  0  0
0  0  0 
0  0  0 
0 15  7 
0  0  0
15  0 15
0  0  0 
0  0  0 
0  0  0
\end{lstlisting}
\caption[short]{Another P3 PPM but formatted in a different manner}
\end{figure}

\footnotetext[1]{
Note that, a whitespace is any of the following characters:
\begin{itemize}
    \item Newline ($\backslash n$, also known as a LF).
    \item Tabs (vertical $\backslash v$ and horizontal $\backslash t$).
    \item Carriage Return ($\backslash r$).
    \item Form Freed ($\backslash f$).
    \item Space
\end{itemize}
All characters except the Line Feed character are a "white space"
Together, all of these are known as "whitespace" characters.
}

\end{document}